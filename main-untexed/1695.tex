1 \leq k < n
